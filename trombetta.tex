\documentclass[10pt]{article}

\title{Running biomechanics and the effect of heel versus forefoot strike}
\author{John Trombetta\thanks{Author is with the Department of Mechanical Engineering at the United States Naval Academy. Address for correspondence: \emph{m236348@usna.edu}}}
\date{\today}

% todo 
% add force sensor dimensions in metric, and to force sensor drawing legend
% fill in trombetta shoe make and model, how was he trained to do forefoot strike
% add camera distance
% how was camera mounted / who filmed / whose treadmill / add to acknowledgements? 

\usepackage[separate-uncertainty=true,multi-part-units=single]{siunitx}
\DeclareSIUnit{\year}{y}
\DeclareSIUnit{\inch}{in}
\DeclareSIUnit{\foot}{ft}
\DeclareSIUnit{\poundforce}{lbf}
\DeclareSIUnit{\pound}{lb}
\DeclareSIUnit{\frame}{frame}
\usepackage{graphicx}
\usepackage[round,authoryear]{natbib}
\bibliographystyle{apalike}
\usepackage[plain]{fancyref}
\usepackage{listings}
\usepackage{amsmath,amsfonts,amssymb}
\usepackage{booktabs}
\lstset{%
  basicstyle=\ttfamily,
  columns=fullflexible,
  showstringspaces=false}
\usepackage[dvipsnames,svgnames]{xcolor}
\usepackage{hyperref}
\hypersetup{%
  colorlinks=true,
  linkcolor=violet,
  urlcolor=blue,
  citecolor=blue}
\usepackage{svg}
%\usepackage{svg-extract}
\usepackage{fullpage}
\newcommand{\Homosapiens}{\emph{Homo sapiens}}
\newcommand{\Hsapiens}{\emph{H.~sapiens}}
\newcommand{\Matlab}{Matlab}




\begin{document}
\maketitle
\begin{abstract}
The goal of my research was to study the effectiveness and injury risk of heel strike versus forefoot strike running. Medical literature suggests adoption of forefoot strike running can reduce risk of injury, specifically, shin splints, and I wished to examine why. Due to the global COVID-19 pandemic, I improvised a flexure-based force plate kymograph to provide crude measurements of vertical ground reaction force (GRF) during each running style. I also used a smartphone as a wearable accelerometer to obtain body accelerations during each running style. Finally, I used a smartphone camera to perform kinematic analysis of each running style while running on a treadmill. I found that while the two running styles did not appear to have grossly different overall vertical forces and accelerations, the kinematics data suggest fine scale differences, especially around the instant of contact with the ground, that could affect compressive loads in the lower leg. Forefoot strike may produce larger horizontal ground reaction forces on pushoff, while heel strikes may produce larger, more instantaneous impacts that have higher potential to cause injury, compared to a more spread out force enabled by compliant gaits like a forefoot strike.  
\end{abstract}
{\scriptsize\textbf{Keywords: }shin splints, running, kinematics, ground reaction forces, kymograph}

\section{Introduction}
% Trombetta: Most of what I have is not running shin splints stuff; hopeful you found some there. I do have some general human running stuff in the list. You might ask 1/C Anmol Walha about running biomechanics, and 1/C Lily Bautista about walking and ankles? I have their 485 projects but not the references they found... 

To study the effectiveness and injury risk of heel strike versus forefoot strike running.

Explain heel strike versus forefoot strike

Work from an inverted triangle (broader topics to more specific). Explain what you are interested, review some relevant literature, and then set up what your specific research question is. Cite literature using the author-year format, as in \citep{buck2020go}. If you need pictures to explain the relevant biomechanics, feel free to include. This section should also say a little why your research matters. 

The last part of this section should be the specific hypotheses you seek to test. 
 %\section{Introduction}
\section{Methods and materials}
\label{sec:methods}

\subsection{Experimental subject}
I used a single experimental subject, \SI{19}{\year} old male, height \SI{5}{\foot} \SI{8}{\inch} (\SI{1.73}{\meter}), mass \SI{160}{\pound} (\SI{72.6}{\kilo\gram}) \textbf{FIX}. For this subject, measured hip height from the ground was \SI{0.88}{\meter} \textbf{FIX}. For scale purposes, the subject shoe size was US9.5M, heel to toe length \SI{0.30}{\meter} \textbf{FIX}. During running trials, the subject wore physical training (PT) gear consisting of running shorts, a tshirt, and athletic shoes (Lunarlite; Nike, Beaverton, OR) \textbf{FIX} with high contrast dark uppers and a lite sole and heel. The subject was in good physical condition, having completed one year of training as a midshipman at the US Naval Academy. The subject provided his informed consent before measurements\footnote{This pilot study was conducted under the following exemption: The project in EW282D is designed to teach research methods through student interaction with data about individuals. Student class assignments typically do not meet the federal regulatory definition of research, thus do not require IRB application, approval, or oversight.}.  

\subsection{Improvised force plate kymograph}

\subsection{Measurement of accelerations using a smart phone}

\subsection{Video kinematics on a treadmill}
 %\section{Methods and materials}
\section{Results}
\label{sec:results}
%Explain factually what you found... leave interpretation of what it means for the final discussion section. Here you would include plots of what you found or comparison tables.

\subsection{Improvised force plate kymograph}
Heel versus toe-strike two trials. Toe-strike only shows lateral movement of rig, primarily with left foot. 
\begin{figure}
\begin{center}
\textbf{Fill in from slides 11 and 12.}
\end{center}
\caption{Improvised force plate recordings of vertical ground reaction force; (a) and (b) trials comparing heel strike and toe strike; (c) checking toe strike.}
\label{fig:results:forceplate}
\end{figure}

\subsection{Measurements of acceleration using a smartphone}

\subsection{Video kinematics on a treadmill}
 %\section{Results}
\section{Discussion}
%Here discuss what your results mean. Are your hypotheses supported or not? Do you have an answer to your overall research question?

The data collected provides some evidence that forefoot striking produces a larger force upon pushoff. 

The data in other studies show that heel strikes produce a larger more instantaneous impact causing injuries compared to a slower more spread out force with a forefoot strike. 
 %\section{Discussion}

\section{Acknowledgements}
I thank Cameron Smith, Alexis Pak, and Sofia Figueroa for their assistance as classmates in EW282D / EW496. I also thank my father, Nick Trombetta, for lending use of a treadmill and for assistance with filming video kinematics. USNA Biomechanics is supported by Lockheed Martin. 

% References
\bibliography{trombetta.bib}

%\clearpage
%\appendix
%\renewcommand{\figurename}{Supplementary Figure}
%\renewcommand{\thefigure}{S\arabic{figure}}
%\input{method-details.tex}
\end{document}
